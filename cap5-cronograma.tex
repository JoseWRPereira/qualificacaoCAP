%\chapter{Cronograma}

As atividades a serem desenvolvidas para a 
conclusão e defesa da dissertação de mestrado 
são apresentadas na forma de descrição e na forma de tabela.

A descrição apresenta a série de tarefas 
a serem realizadas após a qualificação, 
sendo cada uma das tarefas denotada como uma entrega 
a ser realizada ao orientador.

O cronograma ajustado para datas de entrega dos fragmentos do trabalho,
 possibilita a visão e o esforço focados na tarefa, 
e foi baseado em metodologias ágeis.


As tarefas são as seguintes:

\begin{enumerate}
  \item \label{LPA2v}	    Entregar o estudo da LPA2v aplicada ao Controle de Sistemas.

A LPA2v possui uma grande diversidade de aplicações, 
e tendo o controle de sistemas 
uma das áreas de pouca ou talvez nenhuma aplicação, 
pois ainda não foi encontrado qualquer artigo que mostrasse seu uso. 
Assim essa tarefa consiste em se aprofundar nas pesquisas 
da utilização da LPA2v em sistemas de controle de processos. 

  \item \label{controlar}   Entregar a implementação de um controlador utilizando LPA2v.

Implementar dois modelos de controlador, 
baseados no exemplo de aplicação em controle de sistemas
trazido pelo Profº Dr. João Inácio da Silva Filho em sua 
tese de doutorado defendida na Universidade de São Paulo, 
sendo ele um dos precursores da aplicação da LPA2v. 

  \item \label{configurar}  Entregar a configuração do controlador.

Formatar um modelo estabelecendo a configuração do controlador desenvolvido.

  \item \label{descrever}   Entregar a descrição do controlador e dos parâmetros de ajuste.

Dado o controle utilizando a LPA2v em operação, 
serão descritos os parâmetros de ajuste e a configuração do controlador, 
permitindo uma fácil reprodução para novos ensaios e aplicações.

  \item \label{otimizar1}   Entregar a primeira otimização dos parâmetros e analise da performance.

Primeira etapa de otimização após a descrição do controlador,
onde possíveis alterações podem ser realizadas, 
com o intuito de corrigir conceitos e 
formas de aplicação, ou ainda, 
um refinamento do controlador.

  \item \label{otimizar2}   Entregar a segunda otimização dos parâmetros e analisar a performance do controlador.

Segunda etapa de otimização para pequenos acertos de 
configuração ou decodificação.

  \item \label{entregar}    Entregar a revisão de toda a dissertação.

Revisão geral da dissertação, envolvendo conferência das referências, formatação no padrão estabelecida pela instituição avaliadora, correções ortográficas e gramaticais.

  \item \label{finalizar}   Entregar a dissertação finalizada.

Finalizar a versão para correção do orientador.

  \item \label{corrigir}    Entregar a correção da dissertação.

Após apontamentos do orientador, correção final.

  \item \label{imprimir}    Entregar a imprissão da dissertação.

Realizar a impressão de quatro vias da dissertação, sendo três vias a serem  enviadas à banca examinadora e uma para próprio uso.

  \item \label{finalApres}  Entregar a apresentação finalizada.

Baseado na dissertação finalizada, concluir a produção da apresentação.

  \item \label{apresentar}  Apresentar a dissertação.

\end{enumerate}


%\begin{table}[htbp]
\begin{table}[t]
\caption{Cronograma de atividades}
\begin{center}
\resizebox{\textwidth}{!}
{
  \begin{tabular}{c|l|l|l|l|l|l|l|l|l|l|l|l }
  \hline
  \multicolumn{1}{c|}
     {\multirow{2}{*}{Tarefas}} & 
     \multicolumn{2}{c|}{Jul} &
     \multicolumn{2}{|c|}{Ago} &
     \multicolumn{2}{|c|}{Set} & 
     \multicolumn{2}{|c|}{Out} &
     \multicolumn{2}{|c|}{Nov} &
     \multicolumn{2}{|c }{Dez} \\ \cline{2-13}
     \multicolumn{1}{ c|}{} & 12 & 26 & 16 & 30 & 13 & 27 & 11 & 25 & 15 & 29 & 06 & 13 \\ 
  \hline
  \hline
  %\rowcolor[HTML]{EFEFEF}
  \ref{LPA2v} 	  & \cellcolor{lightgray} &&&&&&&&&&&  \\ \hline
  \ref{controlar} && \cellcolor{lightgray} &&&&&&&&&&  \\ \hline
  \ref{configurar}&&& \cellcolor{lightgray} &&&&&&&&&  \\ \hline
  \ref{descrever} &&&& \cellcolor{lightgray} &&&&&&&&  \\ \hline
  \ref{otimizar1} &&&&& \cellcolor{gray} &&&&&&&  \\ \hline
  \ref{otimizar2} &&&&&& \cellcolor{gray} &&&&&&  \\ \hline
  \ref{entregar}  &&&&&&& \cellcolor{darkgray} &&&&&  \\ \hline
  \ref{finalizar} &&&&&&&& \cellcolor{darkgray} &&&&  \\ \hline
  \ref{corrigir}  &&&&&&&&& \cellcolor{darkgray} &&&  \\ \hline
  \ref{imprimir}  &&&&&&&&&& \cellcolor{darkgray} &&  \\ \hline
  \ref{finalApres}&&&&&&&&&& \cellcolor{darkgray} &&  \\ \hline
  \ref{apresentar}&&&&&&&&&&& \cellcolor{black} &  \\ \hline
  \hline
  \end{tabular}
}
\end{center}
\label{tab:cronograma}
\end{table}


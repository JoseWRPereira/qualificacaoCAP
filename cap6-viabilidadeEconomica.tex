Para a efetiva execução da pesquisa, 
dentro das perspectivas propostas e estabelecidas pelo autor, 
sendo  necessária a construção do sistema físico, 
e juntamente com outros recursos,
apresenta-se aqui o aspecto econômico 
referente aos custos envolvidos na viabilização deste trabalho. 
 

\begin{table}[h]
\centering
\caption{Custo dos itens adquiridos para montagem do projeto}
\label{tab:custos}
\begin{tabular}{c|c|c}
\hline
Item  & Descrição  & Valor \\ \hline
\hline
1 & Placa de desenvolvimento modelo Tiva$ ^{TM}$ TM4C123GH6PM & $R\$ 42,00 $ \\ \hline
2 & Placa padrão perfurada 10x15 cm & $R\$ 15,00 $ \\ \hline
3 & Componentes eletrônicos diversos & $R\$20,00$ \\ \hline
4 & Fonte de alimentação & $R\$60,00 $ \\ \hline
\hline
  & Total & $R\$137,00 $ \\ \hline
\hline
\end{tabular}
\end{table}

O item 3 da Tabela \ref{tab:custos} refere-se aos componentes para montagem do circuito de acionamento do motor e como elementos principais pode-se citar: Transistor MOSFET IRF640N, Optoacoplador 4n25, Interruptor óptico HOA0862-T55, capacitores, resistores, diodos, regulador de tensão, conectores. 


Alguns equipamentos ou componentes não geraram custos ao projeto devido a serem itens de uso comum a outras atividades do autor. 

\begin{table}[h]
\centering
\caption{Itens que não geraram custo direto ao projeto}
\label{tab:equipamentos}
\begin{tabular}{c|c}
\hline
Item  & Descrição \\ \hline
\hline
1 & Microcomputador portátil - Notebook \\ \hline
2 & Softwares  \\ \hline
3 & Multímetro \\ \hline
4 & Motor DC \\ \hline
5 & Disco acoplado ao motor \\ \hline
\hline
\end{tabular}
\end{table}

O item 2 da Tabela \ref{tab:equipamentos} 
refere-se aos softwares utilizados em todo o desenvolvimento do trabalho, 
sendo estes ferramentas de uso livre utilizadas previamente pelo autor, 
como sistema operacional GNU/Linux Debian 8(Jessie), 
GNOME Shell, 
Editor de texto e códigos fonte VIM, 
compilador GCC para ARM (arm-none-eabi-gcc), 
GNU Make, 
processador de texto \LaTeX - pdfTEX, 
pacotes geradores de figuras TikZ, PGF e GNU pic(Groff), 
gerador de gráficos GNUPlot, 
teminal de comunicação Minicom e 
gravador LM4Flash.


Todos os custos referentes ao projeto foram custeados pelo próprio autor, em função da natureza e limitação do trabalho proposto.

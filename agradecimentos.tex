À minha família e à minha noiva Fernanda, que além de apoio incondicional souberam compreender todo o esforço e dedicação destinados a este estudo.

Ao Profº Me. Vander Célio Nunes por apresentar e ensinar a Lógica Paraconsistente Anotada Com Anotação de Dois Valore (LPA2v), pela dedicação e orientação ao longo do curso e sempre que precisei.

Ao Profº Me. Rudson de Lima Silva pela dedicação nas aulas e na coorientação, pelo respeito ao conhecimento e pela paixão pela docência, certamente um dos pilares que orientam minha carreira docente.

Ao Profº Engº Erineu Claudemir Bellini pela dedicação e pelo amor ao saber, por servir de exemplo e inspiração desde os tempos de curso técnico nesta mesma instituição uma década atrás, talvez o primeiro farol em que me orientei quando tornei-me instrutor, e tive a oportunidade de perceber com alegria o quanto fui influenciado. 

Ao Profº Me. Leandro Ploni Dantas pela qualidade e notório conhecimento na área de sistemas microcontrolados, objetivo principal pelo qual busquei este curso de especialização, que superou em muito minhas espectativas.

Ao Serviço Nacional de Aprendizagem Industrial (SENAI) de São Paulo, onde ministro aulas como Instrutor de Formação Profissional na área de eletrônica aos cursos de aprendizagem industrial e técnico em eletroeletrônica, pela bolsa de estudos concedida sob aprovação do senhor diretor Profº Me. Carlos Alberto Gomes da unidade SENAI "Frederico Jacob" no Tatuapé e também ao senhor diretor Profº Augusto Lins de Albuquerque Neto, da Faculdade de Tecnologia SENAI Anchieta, coordenadores, professores e demais funcionários dessa unidade, um destaque ao Coordenador Profº Me. Marcos Antônio Felizola que sempre esteve disposto a dialogar e dirimir eventuais problemas e dificuldades.

À todos os colegas que fizeram parte desta jornada algo agradável e divertido, mostrando a individualidade e o potencial de cada um, ampliando a noção de respeito, parceria e amizade. 

 
